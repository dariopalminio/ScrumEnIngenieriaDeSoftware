\chapter{Glosario y Acrónimos}

\begin{table}[h]
\begin{tabular}{lllll}
\cline{1-2}
\multicolumn{1}{|l|}{Nombre}   & \multicolumn{1}{l|}{Descripción}  &  &  &  \\ \cline{1-2}
\multicolumn{1}{|l|}{DoD}   & \multicolumn{1}{l|}{Definición de terminado (Definition of Done)}  &  &  &  \\ \cline{1-2}
\multicolumn{1}{|l|}{DoR}   & \multicolumn{1}{l|}{Definición de completitud (Definition of ready)}  &  &  &  \\ \cline{1-2}
\multicolumn{1}{|l|}{DT}   & \multicolumn{1}{l|}{Equipo de desarrollo (Development Team)}  &  &  &  \\ \cline{1-2}
\multicolumn{1}{|l|}{PB}   & \multicolumn{1}{l|}{Backlog de producto (Product Baclog)}  &  &  &  \\ \cline{1-2}
\multicolumn{1}{|l|}{PI}   & \multicolumn{1}{l|}{Incremento de producto (Product Increment)}  &  &  &  \\ \cline{1-2}
\multicolumn{1}{|l|}{PMI}   & \multicolumn{1}{l|}{Instituto de Gestión de Proyectos (Project Management Institute)}  &  &  &  \\ \cline{1-2}
\multicolumn{1}{|l|}{PO}   & \multicolumn{1}{l|}{Dueño del producto (Product Owner)}  &  &  &  \\ \cline{1-2}
\multicolumn{1}{|l|}{ROI}   & \multicolumn{1}{l|}{Retorno de la inversión (Return On Investment)}  &  &  &  \\ \cline{1-2}
\multicolumn{1}{|l|}{SCRUM}   & \multicolumn{1}{l|}{
\begin{tabular}[c]{@{}l@{}}
Es un juego de rugby en el que, por lo general, tres miembros de cada línea \\
se unen opuestos unos a otros con un grupo de dos y un grupo de tres jugadores \\
detrás de ellos, lo que hace un grupo de ocho personas, tres, dos, tres formados \\
en cada lado; el balón se deja entre la línea divisoria de
 ambos grupos, \\
los jugadores se encuentran abrazados y tomados de la cintura de un compañero de \\
equipo y los del frente hombro a hombro con el oponente, y se trata hacer fuerza \\
grupalmente para desplazar al grupo rival y patear la pelota hacia atrás para que \\
un compañero de equipo la tome.\\
\end{tabular}
}  &  &  &  \\ \cline{1-2}
\multicolumn{1}{|l|}{SB}   & \multicolumn{1}{l|}{Backlog de iteración (Sprint Backlog)}  &  &  &  \\ \cline{1-2}
\multicolumn{1}{|l|}{SM}   & \multicolumn{1}{l|}{Facilitador (ScrumMaster)}  &  &  &  \\ \cline{1-2}
\multicolumn{1}{|l|}{SP}   & \multicolumn{1}{l|}{Iteración (Sprint)}  &  &  &  \\ \cline{1-2}

%% Ejemplo de como agregar una fila con saltos de linea
%% \multicolumn{1}{|l|}{Example}   & \multicolumn{1}{l|}{
%% \begin{tabular}[c]{@{}l@{}}
%% texto texto texto texto texto texto texto  \\
%% texto texto texto texto texto texto texto \\
%% \end{tabular}
%% }  &  &  &  \\ \cline{1-2}

                           &                                                          &  &  & 
\end{tabular}
\end{table}



