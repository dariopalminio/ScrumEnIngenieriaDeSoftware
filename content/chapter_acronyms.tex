\chapter{Glosario y Acrónimos}


  \begin{description}    
  
  \item {\textbf{Agile:} Agile, agilismo, ágil o agilidad se refiere -en general- al lineamiento con los valores ágiles expresados en el manifiesto ágil, cimientos que son cualidades que consideramos valiosas o deseables tener en cuenta. A nivel mental representa un enfoque y a nivel social un movimiento.}
  
  \item {\textbf{AM:} El rol Manager Ágil (Agile Manager) puede ser un Agile Project Manager o Administrador Ágil que media entre el equipo y Managers del resto de la organización. Pueden haber tenido experiencia gestionando equipos y proyectos, experiencia como SM, tener certificaciones Certified ScrumMaster (CSM), PMI-ACP u otras. Es un rol ágil, no de Scrum. } 
   
  \item {\textbf{Artifact:} Artefacto o almacén de incisos de trabajo.}  
 
  \item {\textbf{ATDD:}  Acceptance Test Driven Development es un enfoque por el cual las pruebas de aceptación se hacen antes de desarrollar, en forma de escenarios y se automatizan.}
 
  \item {\textbf{BDUF:} Gran diseño al inicio (Big Design Up Front).}  
  
  \item {\textbf{BA:} Analista de Negocio (Business Analyst). Un BA es quien realiza tareas de análisis de negocio que se describen en BABOK Guide. El BA es responsables de descubrir, analizar y sintetizar la información de una variedad de fuentes dentro de una empresa, incluyendo herramientas, procesos, documentación, y los stakeholders. Es responsable del relevamiento de las necesidades reales de los stakeholders, lo cual con frecuencia involucra la investigar y aclarar sus deseos expresados con el fin de determinar los problemas y las causas subyacentes en el dominio del negocio. Ellos juegan un papel importante en la adaptación de las soluciones diseñadas y entregadas según las necesidades de los stakeholders. Las actividades que realizan incluyen: comprensión de las metas de la empresa y sus problemas, análisis de necesidades y soluciones, análisis de la organización (estructura, política y operaciones), la elaboración de estrategias, impulsión del cambio, y facilitación de la colaboración de los stakeholders.}

  \item {\textbf{BDD:} es un enfoque que busca un lenguaje común para unir la parte técnica y la de negocio, y que sea desde ese lenguaje común desde donde arranque el Testing. BDD es como el puente para unir un ATDD con un TDD.}

  \item {\textbf{BE:} Software de bajo nivel o detrás de la interfaz de usuario (Back-End). Su desarrollo especializado sule darse por BE developer (BE Dev).}

  \item {\textbf{Business Owner:} Responsable de los resultados comerciales y técnicos del producto o servicio.}
  
  \item {\textbf{Business Sponsor:} Es quien financia y participa activamente del desarrollo del producto o servicio.}
  
 \item {\textbf{Cycle time:} Tiempo de ciclo es cuando el trabajo real comienza hasta cuándo está listo para entregarse. Tiempo en que dura una fase de trabajo incluyendo su cola de espera. O sea que es la suma del “touch time” o tiempo de trabajo real y el tiempo de cola. Suele ser una columna compuesta en un tablero Kanban.}

 \item {\textbf{DDD:} El diseño guiado por el dominio (domain-driven design), es un enfoque que pone el foco primario del proyecto en el núcleo y la lógica del dominio basando la programación y los diseños complejos en el modelado del dominio del problema.
}

 \item {\textbf{DevOps:} Integración de development (desarrollo) y operations (operaciones), que se refiere a una cultura o movimiento que se centra en la comunicación, colaboración e integración entre desarrolladores de software y los profesionales en las tecnologías de la información (IT).}
 
  \item {\textbf{DoD:} Definición de terminado (Definition of Done). Significa que el trabajo sobre el ítem está completo y está listo para ser subido a operaciones o listo a entregarse.}
  
  \item {\textbf{DoR:} Definición de preparado o completitud de refinamiento (Definition of ready). Significa que el refinamiento sobre el ítem de trabajo está completado y está listo para ser abordado en la planning.}
  
  \item {\textbf{DT:} Equipo de desarrollo (Development Team). Se refiere al equipo de desarrolladores o a cualquier miembro desarrollador del equipo Scrum.}
  
  \item {\textbf{DUF:} Diseño al inicio (Design Up Front).}

 
  \item {\textbf{Ingeniería de Sistemas:} Actividad profesional con un enfoque científico-técnico, interdisciplinario, organizacional y de pensamiento sistémico para la invención y utilización de técnicas dirigidas a la construcción, operación y mantenimiento de sistemas, incluyendo sistemas sociales empresariales, cibernéticos y sistemas hombre-máquina. Los lineamientos generales de esta disciplina pueden ser encontrados en SEBoK (Guide to the Systems Engineering Body of Knowledge).}
 
  \item {\textbf{Ingeniería de Software:} Actividad profesional con enfoque científico-técnico, sistemático, disciplinado y cuantificable para la invención y utilización de técnicas dirigidas a la construcción, operación y mantenimiento de software, de aplicaciones informáticas o a la automatización de información en sistemas hombre-máquina. Los lineamientos generales de esta disciplina pueden ser encontrados en SWEBOK (Guide to the Software Engineering Body of Knowledge).}


  \item {\textbf{FE:} Software de Interfaz de Usuario (Front-End). Su desarrollo especializado sule darse por FE developer (FE Dev), desarrolladores de interfaz gráfica (Dev UI) y diseñadores UI.}
  
  \item {\textbf{Features:} Representan interacciones y acciones del usuario con el sistema. Son funcionalidades que entregan valor de cara al usuario.}
  
  \item {\textbf{Flow Master:} es también llamado Service Delivery Manager, Flow Manager o Delivery Manager y es un facilitador del flujo en un equipo Scrumban o Kanban.}
    
  \item {\textbf{IT:} Tecnología de la información (Information Technology) es la aplicación de ordenadores y equipos informáticos, con frecuencia utilizado en el contexto de la industria de software como el área encargada de brindar soporte de infraestructura y plataformas para el desarrollo de software.}
  
  \item {\textbf{KPI} Indicador clave de rendimiento (Key Performance Indicators).}
  
  \item {\textbf{Métricas} Las métricas son una combinación de atributos cuantificables pertinentes que comunican información relevante acerca de la calidad de nuestros productos y nuestra productividad\footnote{\cite{INCOSE-2005}}. Para la ingeniería del software, las métricas proporcionan una indicación de la calidad de algún tipo de representación del software basadas en un conjunto de medidas indirectas\footnote{\cite{Roger-Pressman-2002}}. O sea que las mismas están relacionadas a unidades de medida e indicadores que ayudan a medir, descubrir y tomar decisiones para mejorar, corregir problemas, hacer estimaciones, controlar calidad, evaluar productividad y hacer control de proyectos. Intentar medir para mejorar nuestra comprensión de entidades particulares es tan poderoso en ciencia, en ingeniería de software como en cualquier disciplina.}
  
  \item {\textbf{MMP:} Producto comerciable mínimo (Minimal Marketable Product).}
  
  \item {\textbf{MVP:} Producto viable mínimo (Minimal Viable Product).}
  
  \item {\textbf{PB:} Backlog de producto (Product Backlog).}
  
  \item {\textbf{PBI:} Ítem de Product Backlog Item (story, technical story, technical debt, spike, etc.).}
  
  \item {\textbf{PI:} Incremento de producto (Product Increment), código de software funcional entregado o software en funcionamiento.}
  
  \item {\textbf{PMI:} Instituto de gestión de proyectos (Project Management Institute).}

  \item {\textbf{PMO:} Oficina de gestión de proyectos (Project Management Office).}
  
  \item {\textbf{PSPI:} Incremento de producto potencialmente entregable (Potentially Shippable Product Increment), código de software funcional entregable o software en funcionamiento listo para entregar.}
  
  \item {\textbf{QA:} Aseguramiento de calidad (Quality Assurance) o Rol específico para hacer pruebas de software y asegurar la calidad (Quality Assurance Tester).}
  
  \item {\textbf{Risk:} Riesgo es la posibilidad de un problema o incertidumbre relacionada con un proyecto o PBI de un proyecto, que podría alterar significativamente el resultado del mismo de una manera potencialmente negativa. No tiene ningún impacto actual en el proyecto, pero podría tener un impacto potencial en el futuro.}
  
  \item {\textbf{ROI:} Retorno de la inversión (Return On Investment).}

  \item {\textbf{OPCON:} contingencia operacional o fallo de software en operaciones (Operational Contingency).}

  \item {\textbf{SCRUM:} Es un marco de trabajo. El término fue tomado prestado del rugby. En rugby es un juego en el que, por lo general, tres miembros de cada línea se unen opuestos unos a otros con un grupo de dos y un grupo de tres jugadores detrás de ellos, lo que hace un grupo de ocho personas, tres, dos, tres formados en cada lado; el balón se deja entre la línea divisoria de ambos grupos, los jugadores están abrazados y tomados de la cintura de un compañero de equipo y los del frente hombro a hombro con el oponente, y se trata hacer fuerza grupalmente para desplazar al grupo rival y patear la pelota hacia atrás para que un compañero de equipo la tome.}
  
  \item {\textbf{SB:} Backlog de iteración o Sprint Backlog. Es la pila de trabajo comprometido a desarrollar dentro de un sprint.}
  
  \item {\textbf{Scaling Scrum:} Es cualquier implementación de Scrum donde múltiples Equipos Scrum construyen un producto, múltiples productos relacionadoso o un conjunto de características de un producto en uno o más Sprints (NEXUS, Scrum Profesional a Escala, Lucho Salazar, Versión 3.0.0,  Agiles 2016 en Quito, 6-8 Octubre, 2016).}
  
  \item {\textbf{SM:} Facilitador o ScrumMaster. Suele ser deseable que esté certificado como Certified ScrumMaster (CSM) o equivalente. }

  \item {\textbf{SME:} Experto en alguna materia o disciplina (Subject Matter Expert).}
   
  \item {\textbf{SOA:} Arquitectura Orientada a Servicios (Service-Oriented Architecture).}
  
  \item {\textbf{Spt:} Iteración (Sprint).}
  
  \item {\textbf{SP:} Story Point o puntos de historia.}
  
 \item {\textbf{Throughput:} Número medio de unidades procesadas en un tiempo determinado. Tasa de salida del sistema de trabajo.}
  
  \item {\textbf{UI:} Se refiere al diseño de interfaz de usuario (user interface) o al rol encargado de analizar y diseñar interfaz de usuario.}
  
    \item {\textbf{UX:} Se refiere a la experiencia de usuario de un sistema (User Experience) o al rol encargado de analizar y diseñar dicha experiencia (UXer).}
    
  \item {\textbf{VoC:} Voz del cliente.}
  
  \item {\textbf{VUCA:} El entorno VUCA, se caracteriza por la volatilidad, la incertidumbre, la complejidad y la ambigüedad (Volatility, Uncertainty, Complexity and Ambiguity).}
  
  \item {\textbf{WIP:} Trabajo en curso (Work In Process). Cantidad de ítems de trabajo desarrollándose en paralelo en una fase de desarrollo o en una columna de tablero Kanban.}
  
    
\end{description}
  
