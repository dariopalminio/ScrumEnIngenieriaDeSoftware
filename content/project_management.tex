\chapter{Gestión de proyectos}

Scrum está orientado a productos, pues: "es un marco de trabajo para desarrollar, entregar y mantener productos complejos"\footnote{Scrum Guide 2017 (Scrum.org).}. Sin embargo, en las organizaciones de gestión tradicional, los productos se desarrollan mediante la administración de proyectos. Entonces, bajo ese enfoque orientado a proyectos, Scrum puede verse como un marco ágil de gestión de proyectos para el desarrollo iterativo e incremental de productos, valiéndose de equipos autoorganizados. 

Es en esta vía que en esta sección vine a ofrecer una interpretación y perspectiva comparativa y conciliadora con un enfoque de proyectos, no dejando de recordar que Scrum (según la Scrum Guide 2017) no trata sobre proyectos. Es el PMI con la Agile Alliance quienes formalizan la idea de Agilidad y el uso de Scrum en la gestión de proyectos de alta incertidumbre, mediante la Agile Practice Guide\footnote{Agile Practice Guide, PMI and Agile Alliance, 2017.}.

\subsection{Proyecto Scrum}

Un proyecto, en ingeniería de software, es un esfuerzo temporal que se lleva a cabo para crear un sistema, software o resultado único\footnote{Se parafrasea la definición del PMBOOK \cite{PMBOK-2004}.}. Los proyectos son organizados, en una empresa u organización, por el proceso de administración de proyectos. Según este proceso, el ciclo de vida de los proyectos se puede dividir en tres fases: inicio, ejecución y cierre (ver fig. \ref{fig:PMIProject}). Con Scrum se puede implementar este proceso en una forma ágil, haciendo 
el inicio de forma lean, iterando el desarrollo en sprints de ejecución hasta llegar a un cierre donde el producto se mantiene, se traspasa o discontinua (ver fig. \ref{fig:PhasesOfAProject}). 

\begin{figure}[h]
  \centering
  \includegraphics[width=0.99\textwidth]{PhasesOfAProject}
  \caption{Ciclo de vida de un proyecto Scrum.}
  \centering
  \label{fig:PhasesOfAProject} %\ref{fig:PhasesOfAProject}
\end{figure}
\FloatBarrier % Command to control the position of floating images. With its, I can get the figures not to be pushed to the end of the document.
% El comando FloatBarrier es usado aqui para que la imagen se clave en este lugar y que no sea acarreada al final del documento.


\subsection{Inicio}

En una organización ágil, en la fase de inicio del proyecto se inicia un producto o servicio mediante dos actividades principales: el descubrimiento (discovery) y la incepción (inception). Tengo entendido que estas etapas no están formalizadas en el marco ágil, aunque en la práctica, y en la mía en particular, sí se suelen emplear y son aceptadas en diferentes corrientes del movimiento ágil. En este contexto, es deseable que el inicio del proyecto alimente al desarrollo con Scrum con un primer backlog y cierta claridad para comenzar con baja incertidumbre el primer sprint. 

\subsubsection{Descubrimiento}

En el desarrollo lean nos debemos enfocar en traer al cliente al proceso de desarrollo del producto para que se construya algo que, además de querer pagarlo, lo quiera y tenga experiencias de usuario memorables. Necesitamos descubrir lo que la gente quiere o necesita o un problema a resolver. El descubrimiento (discovery) se enfoca en iniciar un desarrollo lean y es parte, en la organización, del descubrimiento de iniciativas de proyectos o productos. Pues, es aquí donde nace el proyecto. En una organización lean, donde se implementa el método científico en el negocio mediante lean startup u otro método, es donde nacen las primeras “macro hipótesis” como parte de la iniciativa del proyecto. Es en este momento en donde se asocia, aunque sea temporalmente, alguna “métrica de éxito” (KPI) que nos permita validar la macro hipótesis. En esta etapa se comienza a entender y definir al cliente y se hace una investigación de cliente (customer research) para ello. Para esto se comienza a analizar el viaje del cliente en el producto o servicio (Customer Journey). También se revisan temas de factibilidad técnica. El resultado de esta fase será la entrada para la etapa de inception. En la organización es la etapa donde se genera la definición de la oportunidad de negocio en una evaluación de oportunidad (opportunity assessment) o caso de negocio (business case) que sirve de entrada para evaluación de iniciativas de proyectos (en la PMO si existiera) y presupuestación (en caso que se trabaje bajo presupuestación).

\subsubsection{Incepción}

En la incepción (inception) es donde se responde el por qué, qué y cómo, a alto nivel, para posibilitar el desarrollo posterior. Al inicio se hace el lanzamiento del producto o servicio (Kickoff), aclarando el porqué se hace el producto, el contexto, la visión y buscando lograr el compromiso de los interesados. Se dilucida a alto nivel las necesidades del cliente, las características del producto o servicio a alto nivel, una arquitectura o metáfora del producto a alto nivel, supuestos y estimaciones relativas iniciales. En aproximadamente una semana de trabajo, el equipo va a entender los objetivos del producto, al usuario o cliente objetivo, el alcance y una guía inicial o roadmap para el desarrollo evolutivo. De aquí saldrá el backlog inicial para comenzar a trabajar con Scrum, en la etapa de ejecución del proyecto, con el sprint 1. En resumen, el objetivo principal de la inception es lograr que el equipo converse, defina y entienda, colaborativamente, lo que se va a desarrollar \footnote{\cite{Caroli-2017}}.

\subsection{Planificación y ciclo de vida}

En la administración de proyectos es necesario planificar y dicha actividad se suele hacer en la fase de inicio ("Starting phase" o "Project Start"). Pero, a diferencia de la metodología clásica predictiva (ver figura \ref{fig:PMIProject}) en que la planificación estaba siempre al inicio y el desarrollo en la fase de ejecución, en el marco Scrum la planificación se distribuye durante todo el ciclo de vida del proyecto y en la fase de ejecución se hace el desarrollo iterativo e incremental de productos (por incrementos de producto) en iteraciones cortas, llamadas Sprint, donde cada iteración tiene su respectiva planificación (ver figura \ref{fig:ScrumProject}) y su incremento de producto, en caso de haberlo logrado (idealmente producto integrado y funcionando de cara al usuario o cliente).

\begin{figure}[h]
  \centering
  \includegraphics[width=0.80\textwidth]{ScrumProject}
  \caption{Proyecto Scrum}
  \centering
  \label{fig:ScrumProject} %\ref{fig:ScrumProject}
\end{figure}
\FloatBarrier % Command to control the position of floating images. With its, I can get the figures not to be pushed to the end of the document.
% El comando FloatBarrier es usado aqui para que la imagen se clave en este lugar y que no sea acarreada al final del documento.

En un sentido, es como si un proyecto se dividiera en muchos pequeñitos, donde “cada Sprint puede considerarse un proyecto con un horizonte no mayor de un mes”\footnote{Scrum Guide 2017 (Scrum.org).}. Entonces, podemos decir que en Scrum se piensa en muchos planes periódicos (a corto plazo). Los mismo pueden estar en un plan mayor a largo plazo, aunque de carácter flexible e interactivo. También se puede realizar un plan global de entregables en base a los incrementos de producto estimados. Pero, desde esta perspectiva, hay que considerar que aunque se trabaje con planificaciones, los planes no son contratos a respetar a rajatabla.

\subsubsection{Niveles de planificación}

En resumen se puede decir que se suelen usar tres niveles de planificación (ver fig. \ref{fig:planning_level}), de los cuales dos son prescriptos por Scrum: Daily y Sprint Planning. La Daily es la planificación a corto plazo, diaria. La Sprint Planning es la planificación a mediano plazo donde se planifica la iteración. En la práctica, se puede utilizar una planificación a largo plazo (Long-term Planning) de la cual se puede obtener un Roadmap o un Release Plan. Esta última puede iniciarse al inicio del proyecto y refinarse en los refinamientos o como una actividad parte de la ejecución.

\begin{figure}[h]
  \centering
  \includegraphics[width=0.99\textwidth]{planning_level}
  \caption{Niveles de planificación}
  \centering
  \label{fig:planning_level} %\ref{fig:planning_level}
\end{figure}
\FloatBarrier % Command to control the position of floating images. With its, I can get the figures not to be pushed to the end of the document.
% El comando FloatBarrier es usado aqui para que la imagen se clave en este lugar y que no sea acarreada al final del documento.

\subsection{Plan de ruta}

El plan de ruta o roadmap (hoja de ruta) de un producto es un plan de alto nivel que describe como el producto va a ir evolucionando en el futuro, de algo simple a algo complejo. Nos permite proyectar en el tiempo donde queremos que esté el producto en el futuro. El objetivo del roadmap es ser una especie de borrador para comunicar, mostrar, reflexionar sobre el futuro del producto, alinear la estrategia y guiar la construcción del mismo. Es una herramienta de planificación dinámica e interactiva, no un contrato para trabajar orientados a deadline.

\begin{figure}[h]
  \centering
  \includegraphics[width=0.85\textwidth]{Roadmap_template}
  \caption{Roadmap ejemplo}
  \centering
  \label{fig:Roadmap_template} %\ref{fig:Roadmap_template}
\end{figure}
\FloatBarrier % Command to control the position of floating images. With its, I can get the figures not to be pushed to the end of the document.
% El comando FloatBarrier es usado aqui para que la imagen se clave en este lugar y que no sea acarreada al final del documento.

\subsection{Planificación de entregables}

Con este marco de trabajo no es necesario hacer una entrega final (o "releasing") ya que se pueden hacer entregas paulatinas. En cada Sprint se puede entregar valor según lo planificado dentro del mismo sprint (planning). Además, para hacer diversas entregas intermedias planificadas se puede crear un plan de muy alto nivel para múltiples Sprints durante una planificación de lanzamientos llamado "Release Plan". Este plan de entregables o de lanzamientos es una guía con la que se pretende reflejar las expectativas sobre qué funcionalidad se implementará y cuándo, aproximadamente, se completará \footnote{\cite{Scrum-Institute-2015}}. También sirve como una base para monitorear el progreso dentro del proyecto. Pero siempre hay que considerar que no es un plan equivalente a un plan clásico, los hitos de releases no deberían ser compromisos rígidos y contractuales y, además, el desarrollo del proyecto no debería centrarse en respetar el plan. Por este motivo el plan de lanzamiento no es un plan estático o rígido; pues, se cambia durante todo el proyecto cuando nuevos requerimientos o conocimientos están disponible y, por ejemplo, cuando entradas PBIs en el Scrum Product Backlog cambian y se re-estiman. Por lo tanto, este plan debe ser revisado y actualizado colaborativamente en intervalos regulares, por ejemplo, después de cada Sprint (en refinamientos o reuniones acordadas).

\subsubsection{Release plan}

Un plan de entregas o "Release Plan" ágil es, normalmente, un conjunto de historias de usuario (o épicas) agrupadas por "releases" (versiones del producto) que se ponen a disposición de los usuarios \footnote{Release plan, jmbeas, 2011; Agile Estimating and Planning, Mike Cohn} o agrupadas por Sprints (ver fig. \ref{fig:Release_plan}). En otras palabras, es una planificación a media distancia como una proyección hacia adelante en una serie de sprints \footnote{Release Planning, Retiring the Term but not the Technique, Mike Cohn, 2012.}. Esta planificación es algo valiosa de hacer cuando se usa el marco Scrum, pero no es requerido por el "núcleo Scrum" o el "Scrum originario" \footnote{Gone are Release Planning and the Release Burndown, Ralph Jocham and Henk Jan Huizer in Community Publications, Scrum.org, Saturday, October 01, 2011; Ken Schwaber and Jeff Sutherland Release Updated Scrum Guide, David Bulkin, Infoq.com on Jul 27, 2011}. Se puede utilizar Scrum con éxito sin necesidad de utilizar “Release Planning”.

\begin{figure}[h]
  \centering
  \includegraphics[width=0.85\textwidth]{Release_plan}
  \caption{Release plan en Sprints ejemplo}
  \centering
  \label{fig:Release_plan} %\ref{fig:Release_plan}
\end{figure}
\FloatBarrier % Command to control the position of floating images. With its, I can get the figures not to be pushed to the end of the document.
% El comando FloatBarrier es usado aqui para que la imagen se clave en este lugar y que no sea acarreada al final del documento.

Hay que remarcar que bajo el marco Scrum, si se hace un Release Plan, debería ser un documento minimalista (buscando el principio de simplicidad), pensado para MVPs (producto de mínimo valor) o entregas frecuentes (respetando el valor de software funcionando), abierto a modificaciones constantes (adaptabilidad y desarrollo evolutivo), consensuado con el equipo (transparencia) y desarrollado por el PO en colaboración con el cliente y con el equipo de desarrollo (priorizando la conversación y no la relación contractual).

Luego hay que tener en cuenta que para crearlo se deben tener disponibles las siguientes cosas:

\begin{enumerate}
\item Un Product Backlog priorizado y estimado.
\item La velocidad estimada del Equipo Scrum.
\item Las condiciones de satisfacción (metas para la agenda, el alcance, los recursos) o impacto deseado.
\end{enumerate}

%Triangle of Project Management or iron triangle
\subsection{Triángulo de la gestión de proyectos}

Bajo el marco de Scrum se cambia la idea relacionada al triángulo clásico de la gestión de proyectos (o triángulo de hierro). El compromiso ya no es entre el tiempo, presupuesto y calidad; sino que se basa en el triángulo de: presupuesto (costo), tiempo (agenda) y funcionalidad (alcance) (ver figura \ref{fig:ScrumProjectManagementTriangle}) \footnote{The iron triangle of planning, Atlassian, Tareq Aljaber, 2017.}. Además, tradicionalmente se ha intentado fijar el alcance para negociar y variar el presupuesto y el tiempo. En cambio, desde la perspectiva ágil, se intentan mantener fijos el tiempo y el presupuesto, mientras se varía el alcance\footnote{\cite{Martin-Alaimo-2014}}. Cuando consideramos el presupuesto fijo (costo del equipo), pretendemos mantener al equipo fijo (equipo estable). Pues se desea equipos estables que maduren, se fortalezcan y, en consecuencia, se transformen en equipos de alto rendimiento. Por otro lado, una manera de trabajar con tiempos fijos es que el equipo se comprometa a trabajar en iteraciones fijas (sprints). Además el equipo podría fijar releases o bloques temáticos dado un tiempo. Y esto no es trabajar bajo “deadlines”, ya que el enfoque no es trabajar orientados a fechas, sino a valor de negocio. Entonces nos queda el alcance, descompuesto en ítems de backlog, y la pregunta ahora es: ¿cuánto puedes hacer en este tiempo y bajo un costo fijo para alcanzar los objetivos de negocio?

\begin{figure}[h]
  \centering
  \includegraphics[width=0.60\textwidth]{ScrumProjectManagementTriangle}
  \caption{Triángulo de Gestión de Proyectos Scrum}
  \centering
  \label{fig:ScrumProjectManagementTriangle} %\ref{fig:ScrumProjectManagementTriangle}
\end{figure}

\subsection{Priorización trade-off}

La ventaja de esta aproximación es que en vez de focalizarnos en plazos con fechas de entrega, nos centramos en desarrollar valor. ¿Cual es el máximo valor que podemos desarrollar en un tiempo dado? Aquí entra el concepto de priorización trade-off, que se lleva a cabo en la ejecución del proyecto (o construcción de un producto). Trade-off es un acuerdo en el cual se deja algo fuera para priorizar otra cosa que se desea más. Pues, ante cambios, se prioriza un ítem reemplazando por otro de esfuerzo semejante para cumplir con entregar valor. Si en una iteración llega un cambio rompe fila, entonces vemos qué podemos sacar del sprint con mismo peso de esfuerzo. Lo mismo si queremos cumplir con un objetivo deseado en un tiempo planificado y necesitamos introducir un nuevo trabajo, pues revisaremos qué podemos sacar de esfuerzo semejante y cuyo valor podemos despriorizar en relación al cambio deseado.

Como vemos, aquí el enfoque viró rotundamente en relación a la administración clásica. En vez de asfixiar a los equipos para que lleguen a cumplir con una fecha, por lo general impuesta arbitrariamente, se les permite tener cierta autonomía (con un responsable de negocio) para que prioricen qué valor de negocio pueden entregar, para un objetivo planeado, con colaboración y participación del equipo mismo.

\subsection{Gestión orientada a producto}

En la gestión orientada a producto nos centramos en iniciativas que evolucionan productos. Entendiendo producto como productos software, servicios, recursos compartidos, suscripciones, agencia, aportación de público, etcétera; basado en software y que ofrece algún tipo de experiencia digital que satisface alguna necesidad o deseo o resuelve un problema, por el cual alguien está dispuesto a pagar. Este tipo de productos pasan por etapas de descubrimiento, desarrollo, marketing y operación. Solo que esta secuencia no es estrictamente en una sola secuencia en cascada, sino más bien en múltiples ciclos de aprendizaje y desarrollo iterativo, incremental y evolutivo, a partir de hipótesis, prototipos y MVPs, como a continuación se explica. 

\subsubsection{Producto mínimo viable}

El verdadero objetivo de Scrum es conseguir un producto mínimo viable o MVP en manos de los futuros clientes cuán rápido sea posible y obtener sus comentarios como feedback temprano\footnote{\cite{Jeff-Sutherland-2016}}. MVP es una estrategia para el aprendizaje de forma iterativa sobre sus clientes para poner a prueba las hipótesis fundamentales del negocio \footnote{\cite{Greg-Gehrich-2012}}. Es un incremento de producto, subconjunto del 20\% de features que representan parte del 80\% de valor\footnote{\cite{Jeff-Sutherland-2016}}. Las características o features del MVP representan conceptualmente al producto final completo (aunque no sea el producto final) y se prueba con un grupo de usuarios tempranos o "early adopters". Las pruebas deben ser medibles y se hacen sobre las hipótesis fundamentales del producto como negocio. Cabe aclarar que, originariamente, el MVP es el primer producto mínimo viable en una cadena evolutiva de entregables. 

\begin{figure}[h]
  \centering
  \includegraphics[width=0.70\textwidth]{MVP}
  \caption{El MVP}
  \centering
  \label{fig:MVP} %\ref{fig:MVP}
\end{figure}
\FloatBarrier % Command to control the position of floating images. With its, I can get the figures not to be pushed to the end of the document.
% El comando FloatBarrier es usado aqui para que la imagen se clave en este lugar y que no sea acarreada al final del documento.


\subsubsection{Evolución de un MVP}

Habitualmente, en la práctica y por simplicidad explicativa, hay quienes denominan MPV a cualquier incremento conceptual, funcional y evolutivo del producto como un todo (ver fig. 5.7). La verdad que no importa qué terminología use (Versión, Release, MVP, etc.), lo importante es que todos los implicados tengan la misma comprensión de los conceptos implicados en la evolución del producto.

\begin{figure}[h]
  \centering
  \includegraphics[width=0.80\textwidth]{MVPs}
  \caption{Evolución de una bicicleta}
  \centering
  \label{fig:MVPs} %\ref{fig:MVPs}
\end{figure}
\FloatBarrier % Command to control the position of floating images. With its, I can get the figures not to be pushed to the end of the document.
% El comando FloatBarrier es usado aqui para que la imagen se clave en este lugar y que no sea acarreada al final del documento.

\subsubsection{Oferta mínima viable económicamente}

Por otro lado, hay otro concepto que se maneja en el ámbito de la agilidad, el mínimo producto vendible o MMP (Minimal Marketable Product). El MMP es el producto con el más pequeño posible conjunto de features y que crea la experiencia del usuario deseada, y por lo tanto puede ser comercializado y vendido, tempranamente, con éxito. Hay quienes lo denominan mínimo producto viable económicamente o EMVP, ya que es la oferta mínima viable económicamente en un mercado. A partir de un MMP es que se llega, posteriormente, a lograr el calce con el mercado o Market-Fit, que es la puesta a punto del motor de la propuesta de valor mediante el desarrollo, la venta y la comercialización, impulsando la demanda a los clientes mayoritarios.

\subsubsection{Ciclo de vida del producto}

Resumidamente, la gestión de proyecto se orienta al producto y su ciclo de vida. Un producto puede evolucionar mediante una secuencia de releases, de los cuales algunos entregan valor directamente de cara al público. Después de una etapa de creatividad y prototipado (si el producto es nuevo y/o novedoso), tendremos uno o una secuencia de entregables MVPs hasta llegar a un MMP, que será el mínimo producto que se lanza al mercado como tal. De allí en adelante comienza el crecimiento en escalamiento (Growth) e incremento de funcionalidad hasta alcanzar una madurez (maturity), para finalmente constituir un producto final estable o commodity (ver fig. \ref{fig:product_evolution})\footnote{\cite{Greg-Gehrich-2012}}.

\begin{figure}[h]
  \centering
  \includegraphics[width=0.80\textwidth]{product_evolution}
  \caption{Crecimiento del producto}
  \centering
  \label{fig:product_evolution} %\ref{fig:product_evolution}
\end{figure}
\FloatBarrier % Command to control the position of floating images. With its, I can get the figures not to be pushed to the end of the document.
% El comando FloatBarrier es usado aqui para que la imagen se clave en este lugar y que no sea acarreada al final del documento.

\subsubsection{Llevar trabajo a los equipos}

Al cambiar la perspectiva hacia estar orientados a productos y mantener equipos estables, cambia el enfoque de la gestión de proyectos. Ya no buscamos recursos humanos para proyectos, desarmando y armando equipos en función de proyectos. Lo que se busca hacer, entre otras cosas, es llevar trabajo a los equipos, buscando productos para equipos. Buscamos alinear a equipos en función de objetivos estratégicos. Mantener productos o servicios de valor para objetivos estratégicos. Donde los equipos son activos de la empresa y que buscan siempre maximizar el valor entregado. Esto puede llevar a cambiar toda la dinámica de funcionamiento de una PMO e, inclusive, desarrollar independientemente de presupuestos (tema que queda fuera del alcance de este libro).

\begin{figure}[h]
  \centering
  \includegraphics[width=0.70\textwidth]{Long_term_team}
  \caption{Llevar trabajo a los equipos de larga duración}
  \centering
  \label{fig:Long_term_team} %\ref{fig:Long_term_team}
\end{figure}
\FloatBarrier % Command to control the position of floating images. With its, I can get the figures not to be pushed to the end of the document.
% El comando FloatBarrier es usado aqui para que la imagen se clave en este lugar y que no sea acarreada al final del documento.

\subsection{Gestión de riesgos}

Si bien la gestión de riesgos no es parte de scrum y un facilitador no se encarga de la gestión al modo tradicional como la gestión de riesgos, sí debería velar por mitigar los problemas que surjan en el proyecto y apoyar, en consecuencia, a la gestión de riesgos. En este sentido, no solo se encargaría de ayudar a desbloquear problemas y reducir impedimentos para que el sistema de trabajo fluya, sino que también puede preocuparse de prever issues para anticiparse a los problemas y controlar así, de antemano y en la medida de lo posible, los riesgos asociados a bloqueos del flujo de trabajo que atentan contra los objetivos del proyecto. Lo difícil es lograr una gestión de riesgos ágil en vez de una gestión pesada, dificultosa y que demande mucho esfuerzo de gestión. En esta vía, es preferible mantener un simple registro de riesgos con información concisa. Los datos principales a registrar de un riesgo, además de su nombre, pueden ser: descripción, probabilidad (probability), impacto (severity), criticidad (criticality), acciones de mitigación, dueño y estado.

Algo simple que se puede lograr trabajando con criticidad es usar tres valores en la probabilidad de ocurrencia y en el impacto (1, 2 y 3). El impacto representa la severidad de la ocurrencia de un problema asociado al riesgo o el tiempo perdido (size of loss). Por otro lado, la criticidad representa la prioridad del riesgo o el grado en que ese riesgo afecta negativamente al proyecto (exposure). El mismo será resultado del producto entre la probabilidad y el impacto. En consecuencia, la criticidad podría ser: 1, 2, 3, 4, 6, 9.

\begin{figure}[h]
  \centering
  \includegraphics[width=0.40\textwidth]{RiskProblemDiagram}
  \caption{Diagrama de riesgos, problemas e ítems del proyecto (PBIs).}
  \centering
  \label{fig:RiskProblemDiagram} %\ref{fig:RiskProblemDiagram}
\end{figure}
\FloatBarrier % Command to control the position of floating images. With its, I can get the figures not to be pushed to the end of the document.
% El comando FloatBarrier es usado aqui para que la imagen se clave en este lugar y que no sea acarreada al final del documento.

Las herramientas gráficas que se pueden usar para dar visibilidad de los riesgos son: a) la “matriz de riesgos”; b) el “histórico de criticidad”; c) o el gráfico de curva de riesgo quemado (Risk Burn-down Chart).

La gestión de riesgos no es independiente de la gestión de impedimento o de problemas (issues o problem). Muchos de los problemas surgidos en el proyecto están asociados a riesgos identificados. Por tal motivo, la gestión de riesgos es útil, porque podemos mitigar los problemas de antemano. Por dicha razón, puede ser valioso llevar un registro de los issues y su relación con los riesgos. Siempre sin perder de vista no caer en el énfasis de la documentación ni en el afán de reportar. Si considera necesario no formalizar en registros esta gestión, no lo haga, puede ser un trabajo desperdiciado. Se debe buscar la simplicidad y el foco en el flujo de trabajo sin impedimentos.

\begin{figure}[h]
  \centering
  \includegraphics[width=0.70\textwidth]{RiskFlowDiagram}
  \caption{Estados posibles de un riesgo.}
  \centering
  \label{fig:RiskFlowDiagram} %\ref{fig:RiskFlowDiagram}
\end{figure}
\FloatBarrier % Command to control the position of floating images. With its, I can get the figures not to be pushed to the end of the document.
% El comando FloatBarrier es usado aqui para que la imagen se clave en este lugar y que no sea acarreada al final del documento.
