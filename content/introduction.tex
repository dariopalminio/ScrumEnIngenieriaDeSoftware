\chapter{Introducción}

\section{Historia}

\section{Definición}

Scrum es un marco de trabajo  (framework) para construir y mantener productos complejos \cite{SBOK-2013} \cite{Scrum-Alliance-2015}. 
Scrum funciona como una implementación del ciclo de mejora continua de Deming (PDCA) y como una implementación de los principios ágiles y principios Scrum. 
Scrum no es exactamente un proceso íntegro, metodología completa o una técnica para construir productos; 
sino que, es un marco de trabajo dentro del cual se pueden emplear varias técnicas y procesos \cite{Agile-Atlas-2012}. 

\section{Metodología}

Hay quienes consideran que Scrum no es una metodología, como por ejemplo Ken Schwaber, entre otras cosas porque no especifia exactamente el cómo se hacen las cosas, sino que dice el qué hacer. Sin embrargo, considerando que define roles, artefactos, actividades, flujo del ciclo de actividades Scrum [Agile Atlas 2013], reglas y algunas sugerencias de implementación como, además, al definir el flujo del ciclo de Scrum o flujo de trabajo define parcialmente un cómo, en el cual incluye una secuencia básica de cosas a hacer; por eso, y sin ser puristas, se puede considerar como una forma de metodología de trabajo y de gestión. O sea que Scrum puede funcionar como una metodología a alto nivel o plataforma de trabajo sobre la cual pueden funcionar otras metodologías, más específicas de producción y desarrollo, y otras técnicas y procesos. Por este motivo, Scrum puede ser adaptada a diversas empresas y organizaciones que trabajen con metodologías diversas pero compatibles con los lineamientos de Scrum. Se puede usar Scrum y a su vez utilizar técnicas de otras metodologías para implementar sus actividades y sugerencias. O sea que cuando se usa Scrum se hace una aproximación a Scrum empleando diversas técnicas y posiblemente otras metodologías.

\section{Ámbito de aplicación}

Relacionado a su ámbito de aplicación se pude decir que Scrum no es un marco de trabajo orientado a implementarse en cualquier dominio y contexto. Scrum está pensado para proyectos bajo “dominios complejos” [Snowden 2007] donde existe un grado alto de incertidumbre y baja predictibilidad. O sea que es útil en ámbitos con requisitos inciertos y riesgos técnicos altos. Está orientado a contextos que necesitan niveles altos de creatividad, innovación, interaccion y comunicación. Por este motivo, es bastante empleado en la industria de software, ya que en la misma existen contextos específicos de alta complejidad e incertidumbre con necesidad de creatividad e innovación. Pero también se utiliza en otras industrias con dominios de problemas de complejidad semejante. Por ejemplo ha sido empleado en: educación, organizaciones de campañas publicitarias, industria de productos de innovación, empresas de editoriales de libros, etcétera.

\section{Visión general}

\begin{figure}[h]
  \centering
  \includegraphics[scale=0.5]{ScrumMapMind}
  \caption{Mapa mental sobre Scrum}
  \centering
  \label{fig:ScrumMapMind} %\ref{fig:ScrumMapMind}
\end{figure}
